%!TEX root = ./thesis.tex 
%%%%%%%%%%%%%%%%%% % Font settings %%%%%%%%%%%%%%%%%%%%% 
\usepackage[osf]{newpxtext} % Palatino font with old-style figures
\linespread{1.05}
\usepackage[T1,euler-digits]{eulervm} % Euler-VM fonts for maths 
\usepackage[scaled]{beramono}   % typewriter font
\usepackage[T1]{fontenc}	    % Activate Type 1 fonts 
\usepackage[utf8]{inputenc}     % We want æøå 
\usepackage{sectsty}            % Change section and chapter header 
\usepackage{textcomp}
\usepackage{anyfontsize}
\usepackage[protrusion=true,expansion=true]{microtype}
\usepackage{textcase}

%%%%%%%%%%%%%%%%%% % General packages %%%%%%%%%%%%%%%%%% 
\usepackage[english]{babel}    % language 
\usepackage{xspace} 
\usepackage{verbatim}          % at least use begin{comment} end{comment} 
\usepackage{booktabs}          % Publication quality tables 
\usepackage{graphicx} 
\usepackage{bmpsize}           % Handles figures 
\usepackage{subcaption} 
\usepackage{wrapfig} 
\usepackage[dvipsnames]{xcolor}     % Handles colors 
\usepackage{setspace}               % Easy setting of line spacing Change the default alignment of a image from left or right an easy manner: 
\usepackage[export]{adjustbox} 
\usepackage[final]{pdfpages} 
%\pdfminorversion=7                 % Raise the version of PDF to 1.7 if needed
\usepackage{epstopdf}
\usepackage{multirow,multicol}
\usepackage{array}
% caption for figures
\usepackage{caption}
\captionsetup[figure]{labelfont={bf},labelformat={default},labelsep=colon,name={Fig.}}
\captionsetup[table]{labelfont={bf},labelformat={default},labelsep=colon,name={Table}}
\captionsetup{compatibility=false}
\usepackage{alltt}
\usepackage{forloop}		        % For-loops!
\usepackage[numbers,sort&compress]{natbib} % Sort numerical keys for multiple cites
\usepackage{hypernat}               % Avoid breaking 'backref' option of hyperref package
\usepackage{float}
\usepackage{units}				    % semantically represent numbers with units
\usepackage{url}
\usepackage{titletoc}               % controlling design of Table of Contents too
\usepackage{appendix}
\usepackage[acronym,toc]{glossaries} % nice macros for handling all acronyms in the thesis
\usepackage{etoolbox}
\usepackage{emptypage}              % remove head and foot for empty pages
\usepackage{quotchap}               % add your favorite quotation before chapter
%%%%%%%%%%%
% Tikz/PGF
%%%%%%%%%%%
\usepackage{tikz}
\usetikzlibrary{positioning,shapes.geometric}
\usetikzlibrary{arrows,automata,positioning}
\usetikzlibrary{calc,decorations,decorations.pathmorphing}
\usetikzlibrary{decorations.text}
\usetikzlibrary{matrix}
\usetikzlibrary{fit}
\usetikzlibrary{patterns}
\usetikzlibrary{topaths}

\usepackage{pgf}

%%%%%%%%%%%%%%%%%%%%%%%%%%%%%%%
% Theorems & Math & Algorithms
%%%%%%%%%%%%%%%%%%%%%%%%%%%%%%%
\usepackage{amsthm}
\newtheorem{definition}{Definition}
\usepackage{thmtools}

\usepackage{amsmath}
\usepackage{mathtools}
\usepackage{amssymb}
\usepackage{amsfonts}
\usepackage{latexsym}

\usepackage{algorithm}
\usepackage[noend]{algpseudocode}
\usepackage[inline]{enumitem}

% Font size and spacing for algorithms
\newcommand{\algfontsize}{\footnotesize}
\newcommand{\algcommentfontsize}{\scriptsize}
\newcommand{\algtypefontsize}{\scriptsize}
\newcommand{\algospacing}{\vspace{\baselineskip}}

% New definition: Switch/Case
\algnewcommand\algorithmicswitch{\textbf{switch}}
\algnewcommand\algorithmiccase{\textbf{case}}
% New "environments"
\algdef{SE}[SWITCH]{Switch}{EndSwitch}[1]{\algorithmicswitch\ #1\ \algorithmicdo}{\algorithmicend\ \algorithmicswitch}%
\algdef{SE}[CASE]{Case}{EndCase}[1]{\algorithmiccase\ #1}{\algorithmicend\ \algorithmiccase}%
\algtext*{EndSwitch}%
\algtext*{EndCase}%

\newcommand{\funclabel}[1]{%
    \@bsphack
    \protected@write\@auxout{}{%
        \string\newlabel{#1}{{\jayden@currentfunction}{\thepage}}%
    }%
    \@esphack
}

%%%%%%%%%%%%
% Listings
%%%%%%%%%%%%
\usepackage[procnames]{listings} % beautiful listings
% some basic setups below as an example
\lstdefinestyle{codestyle}{ %
        basicstyle=\ttfamily\small,
        keywordstyle=\color{RoyalBlue},
        stringstyle=\color{magenta},
        morecomment=[l][\color{ForestGreen}]{//},
        morecomment=[s][\color{ForestGreen}]{/*}{*/},
        commentstyle=\color{olive},
        backgroundcolor=\color{white},
        breakatwhitespace=false,
        breaklines=true,
        captionpos=b,
        keepspaces=true,
        frame=single,
        showspaces=false,
        showstringspaces=false,
        showtabs=false,
        tabsize=2,
        columns=fullflexible,
}
\lstset{style=codestyle}
% if need more key words, then use the setup below as an example 
% with the listing in the latex document together at the same place. 
%\lstset{morekeywords={struct, chan, int, func, const, interface, type}}

%%%%%%%%
% Plots
%%%%%%%%
\usepackage{pgfplots}
\usepgfplotslibrary{groupplots}
\pgfplotsset{compat=1.5}

%%%%%%%%%%%%%%%%%%%%%
% comments and notes 
%%%%%%%%%%%%%%%%%%%%%
% For comments:
% Use notes/todonotes for comments
\usepackage{xargs}  % Use more than one optional parameter in a new commands
\usepackage[colorinlistoftodos,prependcaption,textsize=small,textwidth=2cm]{todonotes}
\setlength{\marginparwidth}{2cm}
\newcommandx{\notes}[2][1=]{\todo[linecolor=blue,backgroundcolor=blue!25,bordercolor=blue,#1]{#2}}

% For text high light
\newcommand{\texthighlight}[1]{\textcolor{OliveGreen}{#1}}

%%%%%%%%%%%%%%%%%%%%%%%%%%%%%%%%%%%%%%%%%%
% font settings of table of contents (ToC) 
%%%%%%%%%%%%%%%%%%%%%%%%%%%%%%%%%%%%%%%%%%
\usepackage{tocloft} % controlling the typographic design of the Table of Contents
% ToC setting for parts 
\makeatletter
\let\oldl@part\l@part
\renewcommand*\l@part[2]{\oldl@part{\MakeUppercase{#1}}{#2}}
\makeatother
\renewcommand{\cftpartfont}{\color{Maroon}\bfseries\normalsize}%
\renewcommand{\cftpartpagefont}{\bfseries}%
% ToC setting for chapters
\renewcommand{\cftchappresnum}{\bfseries}%
\renewcommand{\cftchappagefont}{\bfseries}%
% ToC setting for sections
\renewcommand{\cftsecpresnum}{\scshape}
\renewcommand{\cftsecfont}{\normalfont}
\renewcommand{\cftsecpagefont}{\normalfont}
% Toc setting for subsections
\renewcommand{\cftsubsecpresnum}{\scshape}
\renewcommand{\cftsubsecfont}{\normalfont}
\renewcommand{\cftsubsecpagefont}{\normalfont}

%%%%%%%%%%%%%%%%%%%%%%%%%%%%%%%%%%%%%%%%%%%%%%%%%%%%%%%%%%%%%%%%%%%%%%%%%%%%%%%%
% Layout style setup of the chapter-, section-, subsection-, subsubsection- etc.
%%%%%%%%%%%%%%%%%%%%%%%%%%%%%%%%%%%%%%%%%%%%%%%%%%%%%%%%%%%%%%%%%%%%%%%%%%%%%%%%
\usepackage{titlesec}               % controlling title styles 
% Part Style Setup
\DeclareRobustCommand{\spaceduppercaps}[1]{\textls[80]{\scshape\MakeTextUppercase{#1}}}
\titleformat{\part}[display]
{\bfseries\centering\HUGE}%
{\thispagestyle{empty}\partname~\MakeTextUppercase{\thepart}}{1em}%
{\color{Maroon}\spaceduppercaps}[\bigskip\normalfont\normalsize]

% Chapter Style Setup
\newcommand*\HUGE{\Huge}
\newcommand*\chapnamefont{\itshape\color{RoyalBlue}\HUGE\MakeUppercase}
\DeclareFixedFont{\chapterNumber}{U}{eur}{b}{n}{70}
\DeclareRobustCommand{\chapterspaceduppercaps}[1]{\textls[50]{\scshape\MakeTextUppercase{#1}}}
\newcommand*\chaptitlefont{\baselineskip=1.5\baselineskip\fontsize{18}{21.6}\chapterspaceduppercaps}

\newlength\beforechapskip
\newlength\midchapskip
\setlength\midchapskip{\paperwidth}
\addtolength\midchapskip{-\textwidth}
\addtolength\midchapskip{-\oddsidemargin}
\addtolength\midchapskip{-1in}
\setlength\beforechapskip{18mm}
\newcommand\boxsizept{38pt}

\titleformat{\chapter}[display]
  {\normalfont\filleft}
  {{\chapnamefont\chaptertitlename\enspace}%
    \makebox[\boxsizept][l]{\color{RoyalBlue}\hspace{.001em}%
      \resizebox{!}{\beforechapskip}{\chapterNumber\thechapter}%
      \hspace{.001em}%
      \rule{\midchapskip}{\beforechapskip}%
    }%
  }%
  {60pt}
  {\raggedright\chaptitlefont}
  [\normalsize\vspace*{1.2\baselineskip}\titlerule]
\titlespacing*{\chapter}
  {0pt}{20pt}{20pt}

% Section, Subsection, Subsubsection, Paragraph, Subparagraph Styles Setup:                   
% the baselineskip should be 1.2 times the font size  
\setcounter{secnumdepth}{3} % the depth for numbering to subsubsection 
\titleformat*{\section}{\fontsize{16}{19.2}\selectfont\bfseries}
\titleformat*{\subsection}{\fontsize{14}{16.8}\selectfont\itshape\bfseries}
\titleformat*{\subsubsection}{\normalsize\selectfont\itshape\bfseries}
\titleformat*{\paragraph}{\normalsize\scshape\bfseries}
\titleformat*{\subparagraph}{\normalsize\scshape}

%%%%%%%%%%%%%%%%%%%%%%%%%%
% Hyperref Setup
%%%%%%%%%%%%%%%%%%%%%%%%
% The hyperref package allows advanced hyperlinking functionality
%
\definecolor{CTurl}{named}{Maroon}          % Maroon
\definecolor{CTcitation}{named}{OliveGreen} % WebGreen
\definecolor{CTlink}{named}{RoyalBlue}      % RoyalBlue {cmyk}{1, 0.50, 0, 0}

\usepackage[backref,                        % List citing occurences in the References
			colorlinks=true,                % Colored links
            linktocpage=true,
            pageanchor=true,
            citecolor=CTcitation,           % Color of cite links
            linkcolor=CTlink,               % Color of links
            urlcolor=CTurl,                 % Color of urls
            hypertexnames=false,
			]{hyperref}
\newcommand{\aref}[1]{\autoref{#1}} % Enable links

%%%%%%%%%%%%%%%%%%%%%%%%%%
% Page Setup
%%%%%%%%%%%%%%%%%%%%%%%%
%
% Set fancy page header using fancyhdr package
%
\usepackage{fancyhdr}
\pagestyle{fancy}

% Ensure that the chapter and section headings are in lowercase
\setlength{\headheight}{15pt}
\renewcommand{\chaptermark}[1]{\markboth{#1}{}}
\renewcommand{\sectionmark}[1]{\markright{\thesection\ #1}}

% Delete current section for header/footer
\fancyhf{}

% Define header/footer layout
\fancyhead[LE]{\bfseries\leftmark}
\fancyhead[RO]{\bfseries\rightmark}
\renewcommand{\headrulewidth}{0.5pt}
\fancyfoot[LE,RO]{\bfseries\thepage}
% make space for the rule
\fancypagestyle{plain}{
	\fancyhead{}                           %get rid of the headers on plain pages
    \fancyfoot{}
	\renewcommand{\headrulewidth}{0pt}     % and the line
}

% reset counter for each part of the thesis
\usepackage{chngcntr}          % the package must be after package hyperref 
\counterwithin*{chapter}{part} % reset counter for each part

