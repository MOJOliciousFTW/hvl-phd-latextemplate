%!TEX root = ./thesis.tex 
%%%%%%%%%%%%%%%%%% % Font settings %%%%%%%%%%%%%%%%%% 
\usepackage[scaled]{berasans} 
\usepackage[scaled]{beramono} 
\usepackage{tgpagella} 
\usepackage[T1,euler-digits]{eulervm} 
\usepackage[T1]{fontenc}	  % Activate Type 1 fonts 
\usepackage[utf8]{inputenc} % We want æøå 
\usepackage{sectsty}        % Change section and chapter header 


%%%%%%%%%%%%%%%%%% % General packages %%%%%%%%%%%%%%%%%% 
\usepackage[english]{babel}    % language 
\usepackage{xspace} 
\usepackage{verbatim}   % at least use begin{comment} end{comment} 
\usepackage{booktabs}   % Publication quality tables 
\usepackage{graphicx} 
\usepackage{bmpsize}    % Handles figures 
\usepackage{subcaption} 
\usepackage{wrapfig} 
\usepackage[dvipsnames]{xcolor}     % Handles colors 
\usepackage{setspace}   % Easy setting of line spacing Change the default alignment of a image from left or right an easy manner: 
\usepackage[export]{adjustbox} 
\usepackage[final]{pdfpages} 
\pdfminorversion=7      % Raise the version of PDF to 1.7 
\usepackage{epstopdf}
\usepackage{multirow,multicol}
\usepackage{array}
% caption for figures
\usepackage{caption}
\captionsetup[figure]{labelfont={bf},labelformat={default},labelsep=colon,name={Fig.}}
\captionsetup[table]{labelfont={bf},labelformat={default},labelsep=colon,name={Table}}
\captionsetup{compatibility=false}
\usepackage{alltt}
\usepackage{forloop}		       % For-loops!
\usepackage[numbers,sort&compress]{natbib} % Sort numerical keys for multiple cites
\usepackage{hypernat}       % Avoid breaking 'backref' option of hyperref package
\usepackage{float}
\usepackage{units}				  % semantically represent numbers with units
\usepackage{url}
\usepackage{tocloft}        % controlling the typographic design of the Table of Contents
\usepackage[acronym,toc]{glossaries}  % nice macros for handling all acronyms in the thesis
%\usepackage{chapterbib}

%%%%%%%
% Tikz/PGF
%%%%%%%
\usepackage{tikz}
\usetikzlibrary{positioning,shapes.geometric}
\usetikzlibrary{arrows,automata,positioning}
\usetikzlibrary{calc,decorations,decorations.pathmorphing}
\usetikzlibrary{decorations.text}
\usetikzlibrary{matrix}
\usetikzlibrary{fit}
\usetikzlibrary{patterns}
\usetikzlibrary{topaths}

\usepackage{pgf}

%%%%%%%%%%%%
% Theorems & Math & Algorithms
%%%%%%%%%%%%
\usepackage{amsthm}
\newtheorem{definition}{Definition}
\usepackage{thmtools}

\usepackage{amsmath}
\usepackage{mathtools}
\usepackage{amssymb}
\usepackage{amsfonts}
\usepackage{latexsym}

\usepackage{algorithm}
\usepackage[noend]{algpseudocode}
\usepackage[inline]{enumitem}

% Font size and spacing
\newcommand{\algfontsize}{\footnotesize}
\newcommand{\algcommentfontsize}{\scriptsize}
\newcommand{\algtypefontsize}{\scriptsize}
\newcommand{\algospacing}{\vspace{\baselineskip}}

% New definition: Switch/Case
\algnewcommand\algorithmicswitch{\textbf{switch}}
\algnewcommand\algorithmiccase{\textbf{case}}
% New "environments"
\algdef{SE}[SWITCH]{Switch}{EndSwitch}[1]{\algorithmicswitch\ #1\ \algorithmicdo}{\algorithmicend\ \algorithmicswitch}%
\algdef{SE}[CASE]{Case}{EndCase}[1]{\algorithmiccase\ #1}{\algorithmicend\ \algorithmiccase}%
\algtext*{EndSwitch}%
\algtext*{EndCase}%

\newcommand{\funclabel}[1]{%
    \@bsphack
    \protected@write\@auxout{}{%
        \string\newlabel{#1}{{\jayden@currentfunction}{\thepage}}%
    }%
    \@esphack
}

%%%%%%%%%%%%
% Listings
%%%%%%%%%%%%
\usepackage[procnames]{listings} % beautiful listings
% some basic setups below as an example
\lstdefinestyle{codestyle}{ %
        basicstyle=\ttfamily\small,
        keywordstyle=\color{blue},
        stringstyle=\color{magenta},
        morecomment=[l][\color{ForestGreen}]{//},
        morecomment=[s][\color{ForestGreen}]{/*}{*/},
        commentstyle=\color{olive},
        backgroundcolor=\color{white},
        breakatwhitespace=false,
        breaklines=true,
        captionpos=b,
        keepspaces=true,
        frame=single,
        showspaces=false,
        showstringspaces=false,
        showtabs=false,
        tabsize=2,
        columns=fullflexible,
}
\lstset{style=codestyle}
% if need more key words, then use the setup below as an example 
% with the listing in the latex document together at the same place. 
%\lstset{morekeywords={struct, chan, int, func, const, interface, type}}

%%%%%%%
% Plots
%%%%%%%
\usepackage{pgfplots}
\usepgfplotslibrary{groupplots}
\pgfplotsset{compat=1.5}

%%%%%%%%%%%%
% Other packages we use
%%%%%%%%%%%%

% For comments:
% Use notes/todonotes for comments
\usepackage{xargs}  % Use more than one optional parameter in a new commands
\usepackage[colorinlistoftodos,prependcaption,textsize=small,textwidth=2cm]{todonotes}
\setlength{\marginparwidth}{2cm}
\newcommandx{\notes}[2][1=]{\todo[linecolor=blue,backgroundcolor=blue!25,bordercolor=blue,#1]{#2}}

% Use com for comments
\usepackage{comment}
\newcommand{\com}[1]{
        \mbox{}
       \marginpar{\hrule\footnotesize\raggedright\hspace{0pt}#1\vspace{2mm}}
}


% ToC setting for parts 
\renewcommand{\cftpartfont}{\bfseries\color{Maroon}\large} 
%
%
% Chapter Style Setup:
\usepackage{titlesec}
\newcommand*\HUGE{\Huge}
\newcommand*\chapnamefont{\itshape\color{RoyalBlue}\HUGE\MakeUppercase}
\newcommand*\chapnumfont{\normalfont\color{RoyalBlue}\HUGE}
\newcommand*\chaptitlefont{\normalfont\huge\bfseries}

\newlength\beforechapskip
\newlength\midchapskip
\setlength\midchapskip{\paperwidth}
\addtolength\midchapskip{-\textwidth}
\addtolength\midchapskip{-\oddsidemargin}
\addtolength\midchapskip{-1in}
\setlength\beforechapskip{18mm}

\titleformat{\chapter}[display]
  {\normalfont\filleft}
  {{\chapnamefont\chaptertitlename}%
    \makebox[36pt][l]{\color{RoyalBlue}\hspace{.001em}%
      \resizebox{!}{\beforechapskip}{\chapnumfont\thechapter}%
      \hspace{.001em}%
      \rule{\midchapskip}{\beforechapskip}%
    }%
  }%
  {25pt}
  {\chaptitlefont}
\titlespacing*{\chapter}
  {0pt}{40pt}{40pt}

% Section Style Setup:                                                 
\titleformat*{\section}{\fontsize{18}{20}\bfseries}   
\titleformat*{\subsection}{\fontsize{16}{18}\bfseries}   
\titleformat*{\subsubsection}{\fontsize{14}{16}\bfseries}                  
\titleformat*{\paragraph}{\normalsize\itshape\bfseries}   
\titleformat*{\subparagraph}{\normalsize\scshape\bfseries} 
  
%
% The hyperref package allows advanced hyperlinking functionality
%
\definecolor{CTurl}{named}{Maroon} % Maroon
\definecolor{CTcitation}{named}{OliveGreen} % WebGreen
\definecolor{CTlink}{named}{RoyalBlue} % RoyalBlue {cmyk}{1, 0.50, 0, 0}

\usepackage[%dvipdfmx,        % Use dvipdf driver
			backref,         % List citing occurences in the References
			colorlinks=true, % Colored links
            linktocpage=true,
            pageanchor=true,
            citecolor=CTcitation,  % Color of cite links
            linkcolor=CTlink,  % Color of links
            urlcolor=CTurl,   % Color of urls
			]{hyperref}
%\newcommand{\aref}[1]{\autoref*{#1}} % Prevent links
\newcommand{\aref}[1]{\autoref{#1}} % Enable links

%\usepackage{pdfcomment} % A user-friendly interface to pdf annotations
%
% Set fancy page header using fancyhdr package
%
\usepackage{fancyhdr}
\pagestyle{fancy}

% Ensure that the chapter and section headings are in lowercase
\setlength{\headheight}{15pt}
\renewcommand{\chaptermark}[1]{\markboth{#1}{}}
\renewcommand{\sectionmark}[1]{\markright{\thesection\ #1}}

% Delete current section for header/footer
\fancyhf{}

% Define header/footer layout
\fancyhead[LE,RO]{\bfseries\thepage}
\fancyhead[LO]{\bfseries\rightmark}
\fancyhead[RE]{\bfseries\leftmark}
\renewcommand{\headrulewidth}{0.5pt}

% make space for the rule
\fancypagestyle{plain}{
	\fancyhead{} %get rid of the headers on plain pages
	\renewcommand{\headrulewidth}{0pt} % and the line
}
