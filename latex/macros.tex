%====================================================
%-------------Macro definitions go here--------------
%====================================================

%%%%%%%%%%%%%%%%%%
% Shorthand macros
%%%%%%%%%%%%%%%%%%

\newcommand{\secref}[1]{\S\ref{#1}\xspace}
\newcommand{\figref}[1]{Fig.~\ref{#1}\xspace}
\newcommand{\tabref}[1]{Table~\ref{#1}\xspace}
\newcommand{\algoref}[1]{Algorithm~\ref{#1}\xspace}
\newcommand{\listref}[1]{Listing~\ref{#1}\xspace}
\newcommand{\partref}[1]{Part~\ref{#1}\xspace}
\newcommand{\partsref}[1]{Parts~\ref{#1}\xspace}
\newcommand{\eref}[1]{~(\ref{#1})}
\renewcommand{\equiv}[0]{\ensuremath{:=}}
\newcommand{\etal}{et\,al.}
\newcommand{\paperheader}[2]{\noindent\textbf{Paper #1}: \textit{#2}\\}
\newcommand{\paperitem}[3]{\noindent\textbf{Paper #1}: \textit{#2}\vspace{1em}\\\noindent #3\vspace{2em}}
\newcommand{\tfinal}{\ensuremath{T_{\text{f}}}}
\newcommand{\papernum}[1]{\textbf{#1}}
\newcommand{\eg}{e.g.\xspace}
\newcommand{\ie}{i.e.\xspace}
\newcommand{\cf}{cf.\xspace}

%
% used first time a new concept is introduced
%
\newcommand\concept[1]{{\em #1}}

%
% used to comment out things
%
\newcommand\ignore[1]{}

%
% used for code
%
\newcommand{\smlcode}[1]{{\texttt{#1}}}

% used when referencing to names in figures
\newcommand{\figitem}[1]{\textsf{#1}}

%
% Differentials
%
\newcommand{\tdiff}[2]{\ensuremath{\frac{d#2}{d{#1}}}}
\newcommand{\tdifforder}[3]{\ensuremath{\frac{d^{#2}#3}{d{#1}^{#2}}}}
\newcommand{\pdiff}[2]{\ensuremath{\frac{\partial#2 }{\partial#1}}}
\newcommand{\pdifforder}[3]{\ensuremath{\frac{\partial^{#2}#3}{\partial{#1}^{#2}}}}

%
% Linear algebra
%
\renewcommand{\vec}[1]{\ensuremath{\mathbf{#1}}}
\newcommand{\mat}[1]{\ensuremath{\mathbf{#1}}}
\newcommand{\tildemat}[1]{\ensuremath{\widetilde{\mat{#1}}}}

%
% Theorem environments
%
\newtheorem{definition}{Definition}

%
% Counters
%
\newcounter{ct}

%
% For loop to include papers
%
\newcommand{\includepaperpages}[2]
{
	\forloop{ct}{0}{\value{ct} < #2}
	{
		\begin{figure}[ht]
			\includegraphics[width=.99\textwidth]{#1\the\value{ct}.ps}
		\end{figure}
		\clearpage
	}
}
